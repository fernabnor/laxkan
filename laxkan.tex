\documentclass[10pt,a4paper]{amsart}
\usepackage[subsec,skipmbb,skipfonts]{rune}
\usepackage{laxkanstyle}
%\usepackage{fullpage}

\title[Base change and lax Kan extensions of $(\infty,2)$-categories]{Base change and lax Kan extensions of $(\infty,2)$-categories.}

\author{Fernando Abell\'an}
\address{Norwegian University of Science and Technology (NTNU),
  Trondheim, Norway}

  \date{\today}

  \begin{document}
  \begin{abstract}
    We do basechange and lax kan extensions
  \end{abstract}
  \maketitle 

  \tableofcontents
 

 \section{Preliminaries on fibrations}
 \subsection{Free fibrations}
 \begin{defn}\label{def:freefib}
   Let $\ARplax(\tC)=\FUN^{\plax}([1],\tC)$ be the (op)lax arrow $(\infty,2)$-category of $\tC$ and denote by $\ev_i:\ARplax(\tC) \to \tC$ the functor induced by restriction along the map $\{i\} \to [1]$. We consider a functor 
   \[
     \mathfrak{F}_{(i,j)}: \CATIT_{/\tC} \to \FIB_{(i,j)}(\tC), 
   \]
   whose action on objects is as follows:
   \begin{itemize}
     \item Given $p:\tX \to \tC$ we define $\pi:\mathfrak{F}_{(i,j)}(\tX) \to \tC$ as the pullback
     \[
       \begin{tikzcd}
         \mathfrak{F}_{(i,j)}(\tX) \arrow[r] \arrow[d] &  \tX \arrow[d,"p"] \\
         \AR^{\epsilon\lax}(\tC) \arrow[r,"\ev_{i}"] & \tC
       \end{tikzcd}
     \]
     where the map $\pi$ is induced by $\ev_{1-i}$ and $\epsilon=\op$ if $i\neq j$ and $\epsilon=\emptyset$ otherwise.
   \end{itemize}
   The universal property of the pullback guarantees that this construction extends to a functor of \itcats{}.
 \end{defn}

 \section{Base change}
 \begin{defn}\label{def:fibpattern}
   Let $\tC$ be an \itcat{ }. A \emph{fibrational pattern} $\mathfrak{p}:=(\tC,(i,j),E,L)$ is given by:
   \begin{itemize}
     \item A pair $(i,j)$ where $i,j \in \{0,1\}$ which we call the variance.
     \item A collection of edges $E$ of $\tC$ containing all equivalences.
     \item A collection of 2-simplices $\sigma: [2]  \to \tC$ containing all commutative triangles. 
   \end{itemize}
   Given fibrational patterns $\mathfrak{p}=(\tC,(i,j),E,L)$ and $\mathfrak{q}=(\tD,(i,j),E',L')$ , we say that a functor $f: \tC \to \tD$ is a morphism of fibrational patterns if $f(E)\subseteq E'$ and $f(L)\subseteq L'$.
 \end{defn}



 \begin{ex}
   Given an \itcat{} $\tC$, we denote by $\mathfrak{p}^{(i,j)}_{\flat}:=(\tC,(i,j),\flat,\flat)$ the fibrational pattern with variance $(i,j)$ where the collection of edges is given precisely by the equivalences and the collection $L$ is given by the commuting triangles. Dually, we denote $\mathfrak{p}^{(i,j)}_{\sharp}=(\tC,(i,j),\sharp,\sharp)$ the fibrational pattern where every edge (resp. every triangle) belongs to $E$ (resp. $L$). If the variance is clear from the context we will use the abusive notation $\mathfrak{p}_\flat$ and $\mathfrak{p}_{\sharp}$.
 \end{ex}

 \begin{defn}
   Let $\CATIT_{/(\tC,\mathfrak{p})}$ be the locally full subcategory of $\CATIT_{/\tC}$ whose objects are functors $p:\tX \to \tC$ such that:
   \begin{itemize}
      \item  There exists $i$-cartesian lifts of those 1-morphisms in $E_{\tC}$.\fmarg{Explain this prelim}
      \item There exists $j$-cartesian lifts of 2-morphisms in $L_{\tC}$ which are stable under composition in $\tX$. 
    \end{itemize}
    The morphisms in $\CATIT_{/(\tC,\mathfrak{p})}$ are precisely those which preserve the $i$-cartesian (resp. $j$-cartesian) 1-morphisms (resp. 2-morphisms) above. Given $\tX \to \tC$ and $\tY \to \tC$ in $\CATIT_{/(\tC,\mathfrak{p})}$ we denote by $\MAP_{/(\tC,\mathfrak{p})}(\tX,\tY)$ the mapping $\infty$-category in $\CATIT_{/(\tC,\mathfrak{p})}$.
 \end{defn}

 \begin{remark}
   Observe that $\CATIT_{/(\tC,\mathfrak{p}^{(i,j)}_{\flat})}$ is simply given by the slice $(\infty,2)$-category $\CATIT_{/\tC}$ and that if $\mathfrak{p}=(\tC,(i,j),E,\sharp)$ then $\CATIT_{/(\tC,\mathfrak{p})}$ is given by $\FIB^{\elax}_{(i,j)}(\tC)$.
 \end{remark}

 \begin{remark}
   Let $f:\tC \to \tD$ be functor inducing a map of fibrational patterns $\mathfrak{p} \to \mathfrak{q}$. Then pullback along $f$ induces a functor of \itcats{}
   \[
     f^*:   \CATIT_{/(\tD,\mathfrak{q})} \to \CATIT_{/(\tC,\mathfrak{p})}
   \]
   which we call the base change functor along $f$.
 \end{remark}

 \begin{defn}
   Let $\patp=(\tC,(i,j),E,L)$ be a fibrational pattern and consider a functor $f:\tC \to \tD$. We define the lax basechange functor as the composite
   \[
     f^{\lax}_{\patp}: \CATIT_{/\tD} \to \FIB_{(i,j)}(\tD) \to \CATIT_{/(\tC,\mathfrak{p})}
   \]
   where the first map was given in \cref{def:freefib} and the second map is base change along the map $(\tC,(i,j),E,L) \to (\tD,(i,j),\sharp,\sharp)$ induced by $f$.
 \end{defn}

\begin{defn}
   Let $\patp=(\tC,(i,j),E,L)$ be a fibrational pattern and consider a functor $f:\tC \to \tD$. Then there exists a functor
   \[
      Rf_{*}\colon \CATIT_{/(\tC,\mathfrak{p})} \to \FUN( (\CATIT_{/\tD})^{\op},\CATI), \enspace (\tX \to \tC) \mapsto \MAP_{/(\tC,\patp)}(f^{\lax}_{\patp}(-),\tX).
   \]
\end{defn}

\begin{propn}\label{prop:representable}
  The functor
  \[
     Rf_{*}\colon \CATIT_{/(\tC,\mathfrak{p})} \to \FUN( (\CATIT_{/\tD})^{\op},\CATI)
   \] 
   factors through the composite $\FIB_{(i,j)}(\tD) \to \CATIT_{/\tD} \to \FUN( (\CATIT_{/\tD})^{\op},\CATI)$ where the second functor is the Yoneda embedding.
\end{propn}
\begin{proof}
  ss
\end{proof}

\begin{defn}
  We will call the functor $f_{*}: \CATIT_{/(\tC,\mathfrak{p})} \to \FIB_{(i,j)}(\tD)$ the \emph{fibrational} pushforward functor.
\end{defn}

\begin{thm}
  Let $\patp=(\tC,(i,j),E,L)$ be a fibrational pattern and let $f:\tC \to \tD$ be a functor. Then there exists an adjunction of $(\infty,2)$-categories:
  \[
    f^*: \FIB_{(i,j)}(\tD)  \llra \CATIT_{/(\tC,\mathfrak{p})}: f_*
  \]
\end{thm}

\end{document}
